%intro.tex
FPGA(Field Programmable Gate Array)はユーザによって回路の再構成が可能なLSIであり,目的の処理をハードウェアとして実装可能なデバイスである.
近年FPGAの大容量化,高性能化によって大規模な回路が実現可能になった.これによりFPGAは自動運転を始めとする組込み分野での利用増加が期待されている.

FPGAを用いたハードウェア開発はHDL(Hardware Description Language)によるRTL(Register Transfer Level)設計が広く用いられている.RTLはFPGA上に構成する回路の信号の流れや制御構造を直接設計できる一方,動作検証やデバックが難しく,短期間で複雑な処理の開発は困難である.\cite{bib:fpga}
そのため,FPGAによる開発期間を短縮する方法として,専用ハードディスクとプロセッサを用いたソフトウェアによる処理を組み合わせる方法が考えられる.FPGA上のハードウェアリソースを用いて実装するプロセッサのことをソフトコアプロセッサという.
一般的に組込みシステムではそのコストやサイズ,消費電力などに制限があるためハードウェア資源に制約があることが多く,メモリバンド幅が限られることからメモリシステムの性能が高くないことが多い.メモリシステムの性能が低いと演算性能を高くしてもメモリアクセスに時間がかかり,結果としてシステム全体の性能はメモリシステムの性能によって左右される.\cite{bib:2}
そこで,ソフトコアプロセッサによる処理を考える.すべての処理を専用ハードウェア回路で行うのではなく,高い性能の求められない汎用的な処理についてはソフトコアプロセッサにて行うことにより開発する専用ハードウェア回路を削減でき,ソフトコアプロセッサの性能が向上して専用ハードウェア回路を削減しても求められた性能に達することができればハードウェアの開発コストを抑えることができる.

組込み分野ではAI技術に注目が集まっている.独立行政法人情報処理推進機構の調査によると,将来強化/新たに獲得したい技術として組込み/IoT関連企業の46\%がAI技術を挙げている.\cite{bib:ipa}
