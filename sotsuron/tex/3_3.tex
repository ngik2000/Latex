%3.3.tex

LLVMにおいて特定のターゲットマシン命令の生成は\ref{chp:3_1}で述べたSelectionDAGISelパスによるSelectフェーズにてLLVM IRの命令からターゲットマシン命令に変換されることによって行われる.
この変換はSelectionDAGのノードに対してパターンマッチングを行い,特定のパターンにターゲット命令を対応させる手法で行われる.

このパターンマッチングに用いるパターンはLLVMの固有ドメイン言語であるTableGenによって行われる.TableGenはターゲットマシンの命令やレジスタ等の情報を記述するために用いられる.TableGenでは命令生成のためにパターンの定義だけでなく,アセンブリコード等の出力のためにニーモニックの定義や命令フォーマットの定義も行っている.

SelectionDAGのパターンマッチングの例を図%図番号
に示す.
図%図番号
では加算命令の例を示している.

左からSelectioinDAGの加算命令のノードとそのテキスト表現,命令のパターン定義,変換後のMachineNodeである.

この様にSelectionDAGのパターンと命令ごとに定義されているパターンが一致したときに対応した命令にノードが変換される.
LLVMではRISC-VのV拡張命令のためのパターン定義が既に実装されており,そのパターンと一致した際に変換する命令を独自のベクトル拡張付きRISC-Vの命令に定義しなおすことによってベクトル拡張付きRISC-V命令の生成を実現する.

LLVMにおける命令の定義は命令フォーマットの定義と命令の定義に分かれる.命令フォーマットはTableGenによって命令の種類ごとにフォーマットのクラスを定義を行う.LLVMにおける命令フォーマットは基本クラスRVInstを継承する形で行われる.RVInstでは32ビットのフィールドInstや命令のニーモニックを格納するAsmStringを定義している.このRVInstを継承して異なるフォーマットを定義していく.図%図番号
にRISC-Vの基本命令の命令フォーマットの一つであるR形式の定義を示す.


命令はフォーマットのクラスをインスタンス化する際にニーモニック表記等を定義する.
