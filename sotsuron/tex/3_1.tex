%3.1.tex
LLVMは2000年にイリノイ大学で開発が開始されたコンパイラ基盤である.コンパイラ基盤とはコンパイラに必要となるモジュールをまとめたもので,コンパイラを開発するためのフレームワークである.
LLVMの構成を図%図番号
に示す.LLVMはソースコードをLLVMの中間表現であるLLVM IRに変換するフロントエンド,LLVM IRに対して最適化等の操作やLLVM IRから機械語やアセンブリコードへの変換を行うバックエンドに分かれている.
LLVMではこれらの機能がモジュール化されており,独自機能を実装する以外は既存のものを再利用することができる.例えば新たなアーキテクチャ向けのコンパイラを開発する際はバックエンドのみ実装を行い,フロントエンドについては再利用することができる.

LLVMには既にRISC-Vを対象としたコード生成のためのバックエンドが実装されている.本研究ではRISC-Vを独自にベクトル拡張したベクトル拡張付きRISC-Vの命令の生成を目的としているため,このRISC-V向けのバックエンドに対して変更を加えることによって独自命令の生成を行う.

LLVMバックエンドにおけるコード生成の流れを図%図番号
に示す.
LLVMバックエンドではPassによって処理が行われる.図%図番号
ではLLVMバックエンドにおけるデータフォーマットの変化と実行されるPassを表している.

バックエンドではまずLLVM IRからDAG(Directed Acyclic Graph)であるSelectionDAGへフォーマットを変換する.SelectionDAGはLLVM IRをグラフ形式で表したもので,各命令やデータの依存関係を表現する.SelectionDAGへの変換はSelectionDAGISelパスで行われる.SelectionDAGISelでは