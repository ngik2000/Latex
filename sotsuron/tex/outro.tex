%おわりに

本論文では,スケーラブルなベクトル処理を実現したベクトル拡張付きRISC-V向けのアセンブリコードを得るための手法について検討した.

まずMIQSプロセッサ,ベクトル拡張付きRISC-Vの概要について確認し,コンパイラ基盤であるLLVMにおけるコード生成について確認した.

LLVMでは入力ソースコードをLLVM独自の中間表現であるLLVM IRに変換し,LLVM IRからSelectionDAG,MachineCode,MCLayerを経てアセンブリコードを生成する.LLVMは機能の再利用ができるため,実装済みのRISC-V向けのコンパイラ機能を再利用して我々のベクトル拡張付きRISC-Vのアセンブリコードの生成を目指した.

また,LLVMでは入力ソースコードにおける繰り返し処理をベクトル化されたLLVM IRに変換する自動ベクトル化機能を持っている.ベクトル化されたLLVM IRではベクトル演算の繰り返しと余剰要素の演算によるベクトル処理を行っている.

LLVMバックエンドではドメイン固有言語であるTableGenによって命令やレジスタの情報が定義され,その定義に従ってアセンブリコードなどの生成が行われるため,我々のベクトル拡張付きRISC-Vの命令をLLVMのRISC-V向けバックエンドに定義した.
ベクトル拡張付きRISC-Vの命令の内,プレディケートレジスタを用いないベクトル命令としてベクトル算術論理演算命令であるvadd,vsub,vand,vor,vxorと即値による演算命令であるvaddi,vsubi,vandi,vori,vxori,vsll,vsra,vsrlを実装した.


また,実装した命令のアセンブリコードが正しく生成されるかを実際に配列加算等の命令を入力として生成を検証した.

今後の課題としては,現時点では未実装である命令の実装を行い,動作可能なアセンブリコードを得ることを確認することが挙げられる.
