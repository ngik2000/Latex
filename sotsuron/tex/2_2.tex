%2.2.tex
MIQS(MIPS Instruction processor with Quadword SIMD extension)\cite{bib:hiraishi}
とは,既存のMIPS命令互換プロセッサを元にSIMD演算機能とオンチップメモリを備えたソフトコアプロセッサのことであったが,現在はRISC-Vコアにベクトル演算機能をもたせたソフトコアプロセッサとなっており,特にSIMD拡張を行っているわけではないが名称を引き継ぎMIQSと呼称する.

図%図番号を入れる
にMIQSの全体構成を示す.MIQSのRISC-Vコアプロセッサは既存のRISC-Vプロセッサを使用している.このプロセッサはVerilog-HDLで記述された5段パイプラインプロセッサで回路の複雑化,規模増加を避けるために必要最低限の基本命令セットであるRV32Iに対応している.ベクトル処理ユニットはRISC-V向けのベクトル命令セットに対応している.これによって機械語コードを同時演算数の変更によって作りなおす必要のないスケーラブルなベクトル拡張が実現できた.

命令のフェッチ,デコードはRISC-Vコアプロセッサ上で行い,デコード結果がベクトル拡張命令であったときはベクトル処理ユニットで動作させる.

メモリアクセスはプロセッサからメモリコントローラを介してSDRAMにアクセスする.メモリアクセス要求がプロセッサから発行されたときは,メモリコントローラがSDRAMを制御してメモリアクセスを実現する.また,ベクトルメモリアクセスに関してもメモリコントローラにて行う.
SDRAMコントローラのデータバス幅は128ビットであるから,プロセッサとメモリコントローラ間のデータバス幅は128ビットとなっている.

RISC-Vコアプロセッサは命令フェッチ(IF),命令デコード(ID),実行(EX),メモリアクセス(MA),ライトバック(WB)の5段のパイプラインで構成されている.
ベクトル処理ユニットはベクトル長を256ビット,データの型は32ビット整数型を想定して設計されている.ベクトル処理ユニットとRISC-Vコアの双方で動作する必要のある命令に対応するためベクトル処理ユニットはRISC-Vコアと同じく5段パイプライン構成となっており,RISC-Vコアと強調動作する構成となっている.なお,命令フェッチ部分と命令デコード部分はRISC-Vコアと共通となっている.

MIQSによってスケーラブルなベクトル拡張を実現し,同時演算数の変更によって機械語コードの変更の必要がなくなったが,MIQSが対応している命令セットであるベクトル拡張付きRISC-Vに対応しているコンパイラがないため,ベクトル化アセンブリコードを得ることができない.
そのためコンパイラを作る必要がある.