%2.2.tex
ベクトル拡張付きRISC-Vプロセッサとは,RISC-Vコアにベクトル演算機能を拡張したソフトコアプロセッサである.
図\ref{fig:MIQS_system}
にベクトル拡張付きRISC-Vプロセッサの全体構成を示す.ベクトル拡張付きRISC-VプロセッサのRISC-VコアプロセッサはRISC-Vプロセッサを使用している.このプロセッサはVerilog-HDLで記述された5段パイプラインプロセッサで必要最低限の基本命令セットであるRV32Iで規定された命令を実装している.

\begin{figure}[b]
\begin{center}
    \includegraphics[scale=1.2]{image/MIQS_system.pdf}
    \caption{ベクトル拡張付きRISC-Vプロセッサの構成}
    \label{fig:MIQS_system}
\end{center}
\end{figure}

命令のフェッチおよびデコードはRISC-Vコアプロセッサ上で行い,デコード結果がベクトル拡張命令であったときはベクトル処理ユニットで動作させる.

メモリアクセスに関しては,プロセッサからメモリコントローラを介してSDRAMにアクセスする.メモリアクセス要求がプロセッサから発行されたときは,メモリコントローラがSDRAMを制御してメモリアクセスを実現する.また,ベクトルメモリアクセスに関してもメモリコントローラにて行う.
SDRAMコントローラとメモリコントローラ間のデータバス幅は128ビットであるから,プロセッサとメモリコントローラ間のデータバス幅は128ビットとなっている.

RISC-Vコアプロセッサは命令フェッチ (IF),命令デコード (ID),実行 (EX),メモリアクセス (MA),ライトバック (WB)の5段のパイプラインで構成されている.
%ベクトル処理ユニットはベクトル長を256ビット,データの型は32ビット整数型を想定して設計されている.
ベクトル処理ユニットとRISC-Vコアの双方で動作する必要のある命令に対応するためベクトル処理ユニットはRISC-Vコアと同じく5段パイプライン構成となっており,RISC-Vコアと協調動作する構成となっている.なお,命令フェッチ部分と命令デコード部分はRISC-Vコアと共通となっている.

ベクトル拡張付きRISC-Vプロセッサによってスケーラブルなベクトル拡張を実現し,同時演算数の変更によって機械語コードの変更の必要がなくなったが,ベクトル拡張付きRISC-Vプロセッサが対応している命令セットのアセンブリコード等を出力できるコンパイラがないため,C言語等のソースコードから自動でベクトル化アセンブリコードを得ることができない.
そのためコンパイラを作る必要がある.