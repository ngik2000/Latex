画像処理などの複数のデータに対して同じ演算を行う処理については,単一命令で複数データの処理を行うSIMD命令によるデータ並列処理による高速化が可能である.
しかし,SIMDによるデータ並列処理を行う場合,SIMD命令は1命令で演算するデータ数である同時演算数が決まっている.そのため演算性能を上げるために同時演算数を変更すると機械語コードを作り直す必要がある.
そこで,データ並列処理のために機械語コードの変更なく,同時演算数を変更できるスケーラブルなベクトル拡張を実現したベクトル拡張付きRISC-Vが開発されている.ベクトル拡張付きRISC-Vとは,RISC-Vプロセッサにベクトル処理機能を拡張したものである.ベクトル拡張付きRISC-Vではプレディケートレジスタを用いたベクトル処理を行うことによってスケーラブルなベクトル拡張を実現している.
しかし,このベクトル拡張に対応したコンパイラがないため,ベクトル拡張付きRISC-Vのアセンブリコードを生成できないという問題点がある.これに対する解決策としてベクトル拡張付きRISC-Vに対応したコンパイラの実現方法を検討する.コンパイラ基盤であるLLVMを用いて既に実装済みのRISC-V向けコンパイラの機能を再利用すればコンパイラを一から開発するより容易にアセンブリコードの生成を行うことができる.また,LLVMに備わっている自動ベクトル化機能を用いることでベクトル化されたコードの生成が可能であることから,LLVMを用いることによってベクトル拡張付きRISC-V向けにアセンブリコードを得ることができると考える.

本論文ではLLVMコンパイラ基盤におけるコード生成について述べ,独自命令生成のための命令定義を行う.ベクトル拡張付きRISC-Vの命令の内,ベクトル算術・論理演算命令,即値を用いるシフト命令の実装について述べる.
さらに,実装した命令が正常に入力ソースコードから生成されることを確認する.C言語で記述した配列加算等のプログラムを用いて、そのベクトル化されたアセンブリコードを生成できることを示す.
また,現段階では実装できていない命令として,ベクトルロード・ストア命令等がある.その命令生成に向けて新たに必要な定義や変更点などについて述べる.