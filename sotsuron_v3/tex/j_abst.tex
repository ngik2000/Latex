従来のSIMD命令では同時演算数を変更する場合,それに応じて機械語コードを作り直す手間があった.
そこで,データ並列処理のために機械語コードの変更なく,同時演算数を変更できるスケーラブルなベクトル拡張を実現したベクトル拡張付きRISC-Vが開発された.ベクトル拡張付きRISC-Vではプレディケートレジスタを用いたベクトル処理を行うことによってスケーラブルなベクトル拡張を実現している.
しかし,このベクトル拡張に対応したコンパイラがないため,ベクトル拡張付きRISC-Vのアセンブリコードを生成できないという問題点がある.これに対する解決策としてベクトル拡張付きRISC-Vに対応したコンパイラの開発を検討した.コンパイラ基盤であるLLVMを用いて既に実装済みのRISC-V向けコンパイラの機能を再利用すればコンパイラを一から開発するより容易にアセンブリコードの生成を行うことができる.また,LLVMの機能である自動ベクトル化を用いることでベクトル化されたコードの生成が可能であることから,LLVMを用いることによってベクトル拡張付きRISC-V向けにアセンブリコードを得ることができると考えた.

本論文ではLLVMコンパイラ基盤におけるコード生成について述べ,そのコード生成において用いられる命令の定義手法から,独自命令生成のための命令定義を行った.独自命令の定義として命令フォーマット,出力する命令ニーモニックについて定義を行う.そして,ベクトル化されたLLVM中間表現であるLLVM IRから独自命令の生成を可能としている.

本論文ではベクトル拡張付きRISC-Vの命令の内,ベクトル算術・論理演算命令,即値を用いるシフト命令の実装について述べる.

更に,実装した命令が正常に入力ソースコードから生成されることを確認する.C言語を用いて配列加算等のプログラムを作成し,そのプログラムのアセンブリコードの生成を行う.

また,現段階では実装できていない命令として,ベクトルロード・ストア命令等がある.その命令生成に向けて新たに必要な定義や変更点などについて述べる.