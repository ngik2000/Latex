For processing that performs the same operation on multiple data, such as image processing, it is possible to achieve higher speed by using data parallel processing with SIMD instructions that process multiple data with a single instruction.
However, when using SIMD for data parallel processing, the number of simultaneous operations, which is the number of data to be processed in one instruction, is fixed for SIMD instructions. Therefore, if the number of concurrent operations is changed to improve the processing performance, the machine language code must be rewritten.
To solve this problem, RISC-V with vector extension has been developed for data-parallel processing, which is a scalable vector extension that can change the number of concurrent operations without changing the machine code. RISC-V with vector extension is an extension of RISC-V processor with vector processing capability. The RISC-V with vector extension uses predicate registers for vector processing to achieve scalable vector extension.
However, there is a problem that the assembly code of RISC-V with vector extension cannot be generated because there is no compiler that supports this vector extension. As a solution to this problem, we study how to realize a compiler that supports RISC-V with vector extensions. By using LLVM, which is a compiler infrastructure, and reusing the functions of already implemented compilers for RISC-V, it is easier to generate assembly code than developing a compiler from scratch. In addition, the automatic vectorization function of LLVM can be used to generate vectorized code, so we believe that assembly code for RISC-V with vector extensions can be obtained by using LLVM.

In this paper, we describe the code generation in the LLVM compiler infrastructure and define instructions for generating original instructions. Among the instructions of RISC-V with vector extensions, we describe the implementation of vector arithmetic and logic instructions and shift instructions using immediate values.
In addition, we confirm that the implemented instructions can be successfully generated from the input source code, and show that programs such as array addition written in C can be used to generate their vectorized assembly code.
We will also show that the vectorized assembly code can be generated using programs such as array addition written in C. There are instructions such as vector load/store instructions that have not been implemented at this stage. In this paper, we describe the definitions and changes required to generate such instructions.